\documentclass[12pt]{article}
\usepackage{amsmath}
\usepackage{amssymb}
\usepackage{amsthm}
\usepackage{amscd}
\usepackage{amsfonts}
\usepackage{graphicx}%
\usepackage{fancyhdr}
\usepackage{microtype}
\usepackage[a4paper, total={6in, 10in}]{geometry}


%\addtolength{\topmargin}{-\headheight}
%\addtolength{\topmargin}{-\headsep}
%\setlengt{\hoddsidemargin}{0in}

\pagestyle{fancy}\lhead{Research Summary} \rhead{November 2015}
\chead{{\large{\bf Mateus Borges}}} \lfoot{} \rfoot{\bf \thepage} \cfoot{}

\newcounter{list}

\begin{document}

\newcommand{\PSE}{\texttt{PSE}}
\newcommand{\qCORAL}{\texttt{qCORAL}}
\raisebox{1cm}

My research focuses on improving the scalability and applicability of
Probabilistic Symbolic Execution (\PSE{}). The aim of \PSE{}
\cite{borges2015iterative} is to quantify the probability of reaching
program events of interest assuming that program inputs follow given
probabilistic distributions. The input distributions allow data from
real world observations to be incorporated in the analysis of programs
that interact with their environment, as well as to encode uncertainty
in design assumptions about the usage profile of a program.

The technique has many potential applications, e.g. it can be used in
debugging, for ranking program errors, for quantifying the difference
between two versions of a program, in analyzing the control software
of autonomous vehicles that interact with uncertain environments, for
computing software reliability, and for quantitative information flow
analysis, to name a few. In the past, this kind of analysis was mostly
performed at the model level, limiting its applicability to early
design stages or requiring explicit abstraction from the code. \PSE{}
doesn't share the same limitations and it is able to analyze source
code directly.

\subsection*{Summary of current work}

To compute the probability of an event, \PSE{} needs to quantify the
fraction of the input domain that results in executions leading to the
target event. This operation is one of the main obstacles to the
applicability of \PSE{} in practice: Precise quantification is usually
limited to linear constraints, while only approximate solutions can be
provided in general through statistical approaches. However,
statistical approaches may fail to converge to an acceptable accuracy
within a reasonable time.

To meet this challenge, I developed
\qCORAL\cite{borges2014compositional}, a compositional statistical
approach for the efficient quantification of solution spaces for
arbitrarily complex constraints over bounded floating-point
domains. My PLDI'14 paper presents the compositional strategy used in
\qCORAL: the constraints are split into independent subproblems, which
are analyzed separately, and the results are composed back
together. Partial results from independent subproblems can be computed
faster, and also can be cached and reused. The subproblems are sent to
an off-the-shelf interval constraint solver to break the solution
spaces into even smaller regions, whose union necessarily bounds its
solution space.  \qCORAL then uses stratified sampling, a well known
technique for speeding up the convergence of Monte Carlo simulations,
to analyze the data from the independent regions, and to compose the
results.

My FSE'15 paper \cite{borges2015iterative} further improves \qCORAL
with an iterative distribution-aware sampling approach. This technique
is based in two observations: First, certain paths have higher impact
in driving the program execution toward the occurrence of a target
event. Second, certain paths have higher impact on the convergence
rate of statistical estimation, since they provide more information
about the possibility for the target event to occur. Based on this
insight, we analyze the paths leading to the occurrence of the target
event to rank them according to their impact on the convergence of the
statistical analysis.  Exploiting such ranking, \qCORAL iteratively
re-focuses the sampling to gather more information about the
satisfaction of the most important subproblems first, achieving higher
estimation accuracy in a shorter time. I explored three distinct
ranking strategies: two of them are based on gradient descent
optimization, while the third one uses a simple but efficient
heuristic which gives more importance to the problems whose
probability estimates are farther from convergence.


\subsection*{Future directions}

\noindent \textbf{Domain Coverage and Test Generation} 

\PSE{} builds upon Symbolic Execution, a well-known program analysis
technique that is a fundamental piece of many academic and industrial
tools. Traditional applications of Symbolic Execution are diverse; the
most common are test input generation and error detection. It can be
guided by heuristics to ensure high structural coverage, like branch
coverage or MC-DC.

In a similar way, Probabilistic Symbolic Execution can be used to
produce a set of test inputs that maximize the total percentage of
possible executions covered. I plan to investigate what are the
possible applications of this \textbf{domain coverage} criteria (e.g.
fault detection capabilities).



%% \begin{itemize}
%% \item Investigate what benefits can be obtained by using concolic
%%   execution in place of traditional symbolic execution. Concolic
%%   Execution \textit{... one-liner about concolic execution}
%% \item {something about the new counting library? maybe not...}
  
%% \end{itemize}


%% General problem area
%%  - probabilistic software analysis
%%  - touches multiple areas (smt solving, model counting...)
%%  - many open problems
%% Narrowing focus
%%  - Make ProbSymEx scale to larger programs
%%  - Find applications?
%% Secret weapon?
%%  - Statistical techniques + smart optimizations?
%% Current focus (maybe merge with previous paragraph)
%%  - Adapt concolic execution for probsymex?
%%  - Investigate 
%% End?
%%  - SymEx took 3 decades to become popular
%%  - Reap the benefits of ProbSymEx in 10 years



\bibliographystyle{plain} \bibliography{summary}


\end{document}
