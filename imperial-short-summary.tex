\documentclass[10pt]{article}
\usepackage{amsmath}
\usepackage{amssymb}
\usepackage{amsthm}
\usepackage{amscd}
\usepackage{amsfonts}
\usepackage{graphicx}
\usepackage{fancyhdr}
\usepackage{microtype}
\usepackage{soul}
\usepackage[usenames,dvipsnames]{xcolor}
\usepackage{linegoal}
\usepackage{enumitem}
\usepackage[a4paper, margin=14mm]{geometry}
\usepackage{graphicx}
\usepackage[sort&compress]{natbib}    
\usepackage{verbatim}
\usepackage{titlesec}
\usepackage{bibentry}

%\addtolength{\topmargin}{-\headheight}
%\addtolength{\topmargin}{-\headsep}
%\setlengt{\hoddsidemargin}{0in}

%\pagestyle{fancy}\lhead{Research Summary} \rhead{November 2015}
%\chead{{\large{\bf Mateus Borges}}} \lfoot{} \rfoot{\bf \thepage} \cfoot{}

\newcounter{list}

\renewcommand{\bibfont}{\small}
\setlength{\bibsep}{0ex}


\titlespacing\section{0pt}{5pt plus 1pt minus 2pt}{0pt plus 2pt minus 2pt}

\begin{document}

\newcommand{\PSE}{\texttt{PSE}}
\newcommand{\qCORAL}{\texttt{qCORAL}}
\newcommand{\PC}{\textit{PC}}
%\newcommand{\question}[1]{\colorbox{red}{\parbox[t]{\linegoal}{\textbf{#1}}}}
%\newcommand{\questionDone}[1]{\colorbox{red}{\parbox[t]{\linegoal}{\st{\textbf{#1}}}}}
%\newcommand{\outline}[1]{\colorbox{BurntOrange}{\parbox[t]{\linegoal}{\textbf{#1}}}}
%\newcommand{\outlineDone}[1]{\colorbox{BurntOrange}{\parbox[t]{\linegoal}{\st{\textbf{#1}}}}}
\raisebox{1cm}

\vspace{-20mm}
\section{Introduction}

The pervasiveness of software, the heterogeneity of its users, and the
growing complexity of interactions with third-party components are
introducing a new degree of uncertainty about the execution conditions
of a program, raising the quest for verification techniques able to
deal with and to quantify such uncertainty, both in the problem and in
the verification results \cite{borges2015iterative}. Traditional
qualitative analysis techniques, which can only decide if a property
is satisfied or not, cannot encode this uncertainty.  Thus,
verification of quantitative properties is essential to assure the
safety and reliability of much of the software currently in use.

In the past, verification of quantitative properties has been mostly
performed at model level. Probabilistic model checkers, such as PRISM
\cite{kwiatkowska2011prism}, have been successfully used to analyze
complex systems including network protocols, disk firmware and medical
devices.  However, these techniques require explicit abstraction from
the code, which limits their applicability to early software design
stages or require constant maintenance during development.

Probabilistic Symbolic Execution (\PSE{})
\cite{geldenhuys2012probabilistic,filieri2013reliability} is a recent
technique that aims to bring verification of quantitative properties
to the code level. The objective of \PSE{} is to quantify the
probability of reaching program events of interest assuming that
program inputs follow given probabilistic distributions. The input
distributions allow data from real world observations to be
incorporated in the analysis of programs that interact with their
environment, as well as to encode uncertainty in design assumptions
about the usage profile of a program. The main advantage of \PSE{} is
the lack of need of constructing and maintaining higher level models,
which are often hard to keep consistent with the code, especially for
fast-changing code bases (e.g., those produced by agile development
processes); furthermore, the probabilistic information gathered with
\PSE{} can be used to guide other static and dynamic analysis
techniques, as well as testing. \PSE{} can be used to investigate
quantitative properties of a program behavior, such as: How reliable
is the program under an input distribution, how frequently is this
block executed, and how much of the input space is covered by a test.


%% \section{Background}

%% %% Traditional applications of symbolic execution include test
%% %% case generation and error detection, with many tools available
%% %% \cite{cadar2011symbolic}. Symbolic execution of looping programs may
%% %% result in an infinite symbolic execution tree. For this reason,
%% %% a (user-specified) bound on the search depth is usually imposed.


%% \subsection{Probabilistic Symbolic Execution}
%% %% \begin{figure}[h]
%% %%   \centering
%% %% \includegraphics[width=.5\linewidth]{toolflow}
%% %% \caption{Flow of a generic \PSE{} tool}
%% %% \vspace{-5mm}
%% %% \label{figToolflow}
%% %% \end{figure}

%% Probabilistic symbolic execution (\PSE{}) quantifies the probability
%% of the software satisfying a given (quantitative) property
%% \cite{filieri2013reliability,geldenhuys2012probabilistic,sankaranarayanan2013static,borges2014compositional}.
%% \PSE{} is relevant in contexts where the software is designed to
%% exhibit uncertain or randomized behavior or when the execution
%% environment, including interactions with users, is characterized by a
%% probabilistic profile. In these situations, it is usually more
%% relevant to quantify the probability of satisfying (violating) a given
%% property instead of just assessing the boolean possibility of this
%% occurring.

%% %%%%%
%% \begin{comment}
%% \PSE{} builds upon Symbolic Execution \cite{clarke1976system}, a
%% well-known program analysis technique that executes programs on
%% unspecified inputs by using symbolic inputs instead of concrete
%% values. For each executed program path, the analysis builds a path
%% condition \PC{}, which is essentially the conjunction of boolean
%% conditions characterizing the inputs making the program follow the
%% corresponding execution path. This \PC{} is built according to the
%% branching conditions in the code and it is checked for satisfiability
%% using off-the-shelf solvers. If a \PC{} becomes unsatisfiable it
%% means that the corresponding path is not feasible in the program (and
%% the analysis backtracks). The execution paths followed during the
%% symbolic execution of a program can be composed into a symbolic
%% execution tree, where the nodes represent symbolic program states and
%% the arcs represent transitions between states.
%% \end{comment}
%% %%%%

%% \PSE{} takes as input a program and a usage profile for the program’s
%% input variables. The usage profile includes the domain of input
%% variables and a probability distribution over such domain. The output
%% is an estimate of the probability of satisfying (or violating) a
%% property of interest, e.g., executing a certain block of code or
%% violating an assertion. Internally, \PSE{} uses Symbolic Execution
%% \cite{clarke1976system} to produce a set of \emph{path conditions},
%% which are the conjunction of boolean conditions characterizing the
%% inputs making the program follow the corresponding execution path,
%% that cover the event of interest. In this context, satisfaction of one
%% individual path constraint implies the occurrence of the
%% event. Finally, the solution space of the disjunction of path
%% conditions is quantified (Section \ref{quantification}) and combined
%% with the usage profile to produce the probability of the event to
%% occur.


%% \subsection{Solution Space Quantification}
%% \label{quantification}
%% The quantification of the solution space for a set of constraints is
%% one of the main obstacles to the applicability of \PSE{} in practice.
%% In previous studies \cite{filieri2013reliability,
%%   geldenhuys2012probabilistic}, model counting techniques have been
%% applied to count the number of points within a bounded integer domain
%% that satisfy given linear constraints.  These counts are then coupled
%% with a probabilistic input usage profile, describing how likely is
%% each value in the domain to be an input, to assess the probability for
%% the target event to occur in that usage profile. Borges et
%% al. \cite{borges2014compositional,borges2015iterative} proposed a
%% compositional quantification of the solution space based on Monte
%% Carlo estimation. This approach can deal with arbitrarily complex
%% numeric constraints over floating-point domains with non-uniform
%% distributions, and it speeds up the convergence of the simulation with
%% a focused, iterative sampling algorithm.

%% \begin{comment}
%% Sankaranarayanan et al. \cite{sankaranarayanan2013static} proposed an
%% iterative algorithm to compute tight over-approximating bounds of the
%% actual solution space suitable for efficient volume computation of
%% linear constraints over floating-point domains.
%% \end{comment}

%% Research on quantification of more complex datatypes is ongoing. For
%% strings, the techniques proposed in \cite{luu2014model,
%%   aydin2015automata} can efficiently compute bounds for a subset of
%% the most common string operations.  The support for (bounded) data
%% structures is also in its infancy; in \cite{filieri2015model} a
%% bounded lazy initialization technique is used to explore only the
%% portion of the domain satisfying the constraint. \#SAT and \#SMT
%% solvers can also be applied in boolean and more complex domains (e.g.,
%% requiring reasoning about multiple theories) \cite{biere2009handbook}.

\section{Research Directions}
\label{research}

\PSE{} looks promising as an approach to help developers reasoning
about quantitative properties in their programs. Moreover, the
probabilistic information computed with it can be of use to other
software engineering techniques. However, investigations of the
applicability of \PSE{} are hampered due to the limitations of the
existing implementations. \PSE{} builds upon symbolic execution, and
thus needs to deal with many of its disadvantages, like the
exponential explosion of the number of symbolic execution
paths. Furthermore, it is not always possible to directly use
common optimizations to symbolic execution with \PSE{} due to the need to
quantify the solution space of the paths leading to the target
event. Quantification is also very expensive in general, due to the
inherent hardness of the problem.


%% \begin{itemize}
%% \item \textbf{Path explosion.} Every branch in the program can
%%   potentially create two new paths. Thus, the total number of paths to
%%   be explored is exponential on the number of branches, or even
%%   infinite if the program contains unbounded loops.

%% \item \textbf{Theory support.} Symbolic Execution relies on constraint
%%   solvers to check the feasibility of the explored paths. \PSE further
%%   depends upon model counters to quantify their solution space. Those
%%   tools can handle constraints expressed in specific, well-defined
%%   \textit{theories}, like linear integer arithmetic. Programs that
%%   contain operations that cannot be expressed in the theories
%%   supported by the chosen constraint solver/model counter cannot be
%%   analyzed.

%% \end{itemize}

\vspace{-4mm}
\subsection{Direction 1: Extending and Scaling \PSE{}}

\begin{itemize}
\item \textbf{Compositional Probabilistic Symbolic Execution.} The
path explosion problem inherent to symbolic execution can be
alleviated by computing \textit{summaries} of individual functions
\cite{godefroid2007compositional}. A function summary is a logical
formula that represents all possible pre-conditions over function
inputs and post-conditions over function outputs. Summaries are
generated by computing a pre- and post-condition for each feasible
path of the target function. Once summaries are computed, calls to the
summarized function encountered during the exploration can be replaced
by the summary. Although computing a summary can be expensive, it is
possible to obtain significant gains in performance by carefully
choosing the functions to be summarized at runtime
\cite{anand2008demand}.

\item \textbf{Approximate Model Counting of Complex Datatypes.}
Approximate Model Counters overcome the limitations in terms of
running time and memory demand of exact counting methods, allowing to
deal with much larger problems ($\approx 10\times$ more variables)
\cite{biere2009handbook}.  However, such counters are based on
simulation and can only provide approximate results. While the
accuracy of the results can be arbitrarily improved, the price to be
paid is the increased number of simulations, which increases the
analysis time, possibly making it unreasonably long. I propose to
investigate techniques for approximate model counting of more complex
datatypes, like strings and heap structures. The presence of those
datatypes is widespread among existing software, and supporting them
will enable \PSE{} to be applied more broadly. For strings, a starting
point would be researching how to encode constraints and perform
sampling with probabilistic automata \cite{vidal2005probabilistic}.


%% In previous work\cite{borges2014compositional, borges2015iterative}, I
%% presented \qCORAL, a compositional statistical approach for the
%% efficient quantification of solution spaces for arbitrarily complex
%% constraints over bounded floating-point domains. \qCORAL uses a
%% compositional strategy combined with an off-the-shelf interval
%% constraint solver and a focused, iterative sampling algorithm to
%% improve the speed of the simulation.


%% \item \textbf{Probabilistic Concolic Testing} Concolic Testing
%% \cite{godefroid2005dart} is a very popular approach for test input
%% generation \cite{cadar2011symbolic} that performs both concrete and
%% symbolic execution of the program at the same time. The main advantage
%% of concolic testing is the possibility to simplify \PC s that contain
%% operations (e.g. nonlinear arithmetic) which aren't supported by the
%% constraint solver. This is done by replacing symbolic variables with
%% concrete values obtained during the execution. Performing these
%% simplifications, however, can possibly result in a \textit{incomplete
%%   exploration} of the program due to the loss of symbolic
%% information. On the other hand, the loss of completeness is a better
%% alternative than stopping the exploration when an unsupported
%% constraint is produced. In principle, concolic testing could be used
%% in place of traditional symbolic execution to drive the path
%% exploration stage of \PSE{}. The main benefits of this change are: 1)
%% better support for programs with \PC s that are beyond the
%% capabilities of the chosen constraint solver, and 2) allowing the use
%% of probabilistic information to guide the path exploration during test
%% generation.  In practice, it is necessary to account for the loss of
%% completeness due to simplification: every time the concolic engine
%% simplifies a constraint, there is a risk that paths leading to the
%% desired target event will be missed.
\end{itemize}

\subsection{Direction 2: Applications of \PSE{}}

\begin{itemize}
\item \textbf{Guiding Automated Program Repair with \PSE{}.}
Automated Program Repair \cite{le2015manybugs} (APR) aims to compute,
without manual intervention, a ``patch'' that fixes a defect in a
program. APR approaches can be roughly categorized in two main classes:
search-based and formal.  Search-based approaches generate multiple
candidate solutions according to heuristics which are validated
against a test suite. Formal approaches use formal specifications to
construct a fix for the program which can be mathematically proved to
be correct. I propose to investigate opportunities to use \PSE{} to
improve automatic program repair. One such opportunity is to use
probabilistic information in ranking heuristics results of search-based
approaches. The idea is that candidate solutions with smaller
probabilities of breaking tests will lead more frequently to a correct
patch. Similarly, Formal APR approaches may benefit from a precise
quantification of how many inputs lead to a violation of the formal
specification and how likely those inputs are. Even if a correct patch
is not found, previous research \cite{le2013current} suggests that
patches that do not completely fix the problem may be of utility to
the developer to understand and fix the problem manually.

%% \noindent \textbf{Bug Prioritization} I propose to investigate the
%% application of \PSE{} to ranking bugs. The core idea is that bugs
%% related to failures that happen with a higher probability are more
%% important. This simple heuristic can be used by developers to
%% prioritize their maintenance efforts.

\item \textbf{Generating Tests with High Domain Coverage.}  Structural
  coverage metrics \cite{mockus2009test}, like branch or statemtent
  coverage, are commonly used in the industry to estimate the quality
  of a test suite. The motivation behind it is that the more parts of
  the program are exercised by the suite, the better it is. Maximizing
  coverage is a common target of search heuristics for test input
  generation \cite{godefroid2008automated}. However, structural
  coverage cannot quantify what proportion of the input domain is
  covered, nor which fraction of the user input is represented by the
  test suite. I propose to use \PSE{} to compute these new coverage
  criteria and to evaluate their effectiveness against
  state-of-the-art coverage metrics. If the results are positive, I
  also plan to investigate the use of probabilistic information to
  guide the path exploration during test generation to maximize
  the covered fraction of the input domain.
  
\end{itemize}

\vspace{-4mm}
\section{Collaborations}

%% It is certainly infeasible to solve all current challenges of \PSE{}
%% in two years. However, it is not feasible at this point to choose a
%% few and commit to a hard schedule. External factors, like interest in
%% collaboration from other researchers and availability of case studies,
%% are of great importance to the success of the project, and also
%% difficult to antecipate. Thus, I propose an incremental, iterative
%% strategy to pursue the directions outlined in Section~\ref{research}.
%% This strategy is composed by two dependent tracks:

%% \begin{itemize}
%% \item \textbf{Infrastructure work.} I will produce a suite of tools to
%%   perform \PSE{}, with an focus on ease of use and extensibility to
%%   facilitate the prototyping of research concepts. Planned tools
%%   include \PSE{} engines for Java and C, plus an unified interface for
%%   multiple model counters. The suite must also be able to record a
%%   diverse array of internal information to guide comparative
%%   evaluations, such as explored \PC s, time spent on constraint
%%   solving/solution space quantification, and so on. The aim is to,
%%   once the suite of tools is sufficiently stable, release it to the
%%   community (as opensouce software) to build interest, receive
%%   feedback and enable collaborations. After this initial sprint,
%%   future improvements to the suite will be guided by the needs of the
%%   challenges chosen to be addressed.

%% \item \textbf{Estabilish collaborations.}

  In order to obtain access to case studies and validate results from
  experiments, I plan to estabilish collaborations with key
  researchers from both the academia and the industry. From a more
  pratical point of view, those collaborations will help to identify
  what challenges need to be addressed before exploring a specific
  application of \PSE{}. Here is an initial list:

%  \begin{itemize}
  
  \textit{Dr. Corina Pasareanu, CMU SV/NASA Ames Research
    Center}. Dr. Pasareanu is actively working in enabling \PSE{}
  through the open-souce symbolic execution engine Symbolic
  PathFinder. I collaborated multiple times with Dr. Pasareanu in the
  past, which resulted in four academic publications. Her access to
  NASA systems that could be used as case studies was essential to the
  success of those projects.

  \textit{Prof. Willem Visser, Stellenbosch University}.
  Prof. Visser is one of the authors of the work that introduced \PSE{}
  \cite{geldenhuys2012probabilistic}. He is interested in
  investigating ways to improve the scalability of \PSE{} and extend
  the range of programs that can be analyzed. I also have collaborated
  with him in the past.

  \textit{Dr. Cristian Cadar, Imperial College
    London}. Dr. Cadar is the author of KLEE \cite{cadar2008klee}, a
  well known opensource engine for symbolic execution of C programs.
  One of his main topics of his work is the study of symbolic
  execution-based techniques to improve software reliability and
  security.
    
  \textit{Prof. Alessandro Orso, Georgia Institute of
    Technology}. Prof. Orso work encompasses multiple areas of software
  testing, like test generation, automatic fault localization and
  program repair.
  
%\end{itemize}

\section{(Relevant) Previous Work}

\textbf{\qCORAL{}: Compositional Solution Space Quantification.} To
compute the probability of an event, \PSE{} needs to quantify the
fraction of the input domain that results in executions leading to the
target event. This operation is one of the main obstacles to the
applicability of \PSE{} in practice: Precise quantification is usually
limited to linear constraints, while only approximate solutions can be
provided in general through statistical approaches. However,
statistical approaches may fail to converge to an acceptable accuracy
within a reasonable time. To meet this challenge, I developed
\qCORAL\cite{borges2014compositional,borges2015iterative}, a
compositional statistical approach for the efficient quantification of
solution spaces for arbitrarily complex constraints over bounded
floating-point domains. This approach can deal with arbitrarily
complex numeric constraints over floating-point domains with
non-uniform distributions, and it speeds up the convergence of the
Monte Carlo simulation with a focused, iterative sampling algorithm.



%% \subsection{Schedule}

%% I propose to split the schedule in three distinct phases, listed below.

%% \begin{enumerate}[label=(\texttt{Phase \arabic*})]

%% \item \textbf{Foundations}: \textit{Estimated effort: 6 months}
%%   Develop \PSE engines for Java and C. Write unified interface for
%%   multiple model counters. Release tools to the community to build
%%   interest and obtain feedback.

%% \item \textbf{Research Thrusts}: \textit{Estimated effort: 18 months}
%%   \textit{This phase is composed by three iterations, each one with a
%%     duration of one semester.} Estabilish collaborations with external
%%   researchers.  Decide the research challenge to be tackled in the
%%   current iteration. Estabilish a research plan for the chosen
%%   challenge.  If needed, extend the \PSE{} tools developed in Phase 1
%%   for the realization of experiments. Submit results to relevant
%%   workshops, conferences, and journals.
  
%% \item \textbf{Convergence}: \textit{Estimated effort: 4 months} Enter
%%   ``Writing Thesis'' status. Write thesis. Perform maintenance work on
%%   tools developed during the past phases.

%% \end{enumerate}





\bibliographystyle{plain} \bibliography{summary}

\end{document}
